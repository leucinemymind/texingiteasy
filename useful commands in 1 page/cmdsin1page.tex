%%%%%%%%%%%%%%%%%%%%%%%%%%%%%%%%%%%%%%%%%
% Cheatsheet
% LaTeX Template
% Version 1.0 (12/12/15)
%
% This template has been downloaded from:
% http://www.LaTeXTemplates.com
%
% Original author:
% Michael Müller (https://github.com/cmichi/latex-template-collection) with
% extensive modifications by Vel (vel@LaTeXTemplates.com)
%
% License:
% The MIT License (see included LICENSE file)
%
%%%%%%%%%%%%%%%%%%%%%%%%%%%%%%%%%%%%%%%%%

%----------------------------------------------------------------------------------------
%	PACKAGES AND OTHER DOCUMENT CONFIGURATIONS
%----------------------------------------------------------------------------------------

\documentclass[11pt]{scrartcl} % 11pt font size

\usepackage[utf8]{inputenc} % Required for inputting international characters
\usepackage[T1]{fontenc} % Output font encoding for international characters

\usepackage[margin=0pt, landscape]{geometry} % Page margins and orientation

\usepackage[version=4]{mhchem}

\usepackage{graphicx} % Required for including images

\usepackage{color} % Required for color customization
\definecolor{mygray}{gray}{.75} % Custom color

\usepackage{url} % Required for the \url command to easily display URLs

\usepackage[ % This block contains information used to annotate the PDF
colorlinks=false, 
pdftitle={Cheatsheet}, 
pdfauthor={John Smith}, 
pdfsubject={Compilation of useful shortcuts}, 
pdfkeywords={Random Software, Cheatsheet}
]{hyperref}

\setlength{\unitlength}{1mm} % Set the length that numerical units are measured in
\setlength{\parindent}{0pt} % Stop paragraph indentation

\newcommand{\command}[2]{#1~\hfill{}~#2\\} % Custom command for adding a shorcut

\newcommand{\sectiontitle}[1]{\paragraph{#1} \ \\} % Custom command for subsection titles

\newcommand {\latexcmd}[2]{\texttt{\textbackslash #1\{#2\}}} % commands in LaTex

\newcommand{\environment}[1]{\latexcmd{begin}{#1} \dots \latexcmd{end}{#1}} % environments



%----------------------------------------------------------------------------------------

\begin{document}

\begin{picture}(297,210) % Create a container for the page content

%----------------------------------------------------------------------------------------
%	TITLE SECTION 
%----------------------------------------------------------------------------------------

\put(10,200){ % Position on the page to put the title
\begin{minipage}[t]{210mm} % The size and alignment of the title
\section*{List of useful LaTeX commands} % Title
\end{minipage}
}

%----------------------------------------------------------------------------------------
%	FIRST COLUMN SPECIFICATION
%----------------------------------------------------------------------------------------

\put(5,180){ % Divide the page
\begin{minipage}[t]{90mm} % Create a box to house text

%----------------------------------------------------------------------------------------
%	HEADING ONE
%----------------------------------------------------------------------------------------

\sectiontitle{Preamble essentials}

\command{\latexcmd{documentclass}{<type>}}{Overall layout}
\command{\latexcmd{usepackage}{<pkg>}}{Loads a package}
\command{\latexcmd{title}{your-title}}{Title}
\command{\latexcmd{author}{your-name}}{Author}
\command{\latexcmd{date}{2025-02-31}}{Date}

%----------------------------------------------------------------------------------------
%	HEADING TWO
%----------------------------------------------------------------------------------------				
			
\sectiontitle{Sectioning}
			
\command{\latexcmd{section}{Section 1}}{Numbered section}
\command{\latexcmd{section*}{Section}}{Unnumbered section\footnote{This carries over to subsections and subsubsections as well.}}
\command{\latexcmd{subsection}{Subsection 1.1}}{Subsection}
\command{\latexcmd{subsubsection}{Subsubsection 1.1.1}}{Subsubsection}
\command{\latexcmd{paragraph}{Paragraph}}{Paragraph}
\command{\latexcmd{subparagraph}{Subparagraph}}{Subparagraph}

%----------------------------------------------------------------------------------------
%	HEADING THREE
%----------------------------------------------------------------------------------------	

\sectiontitle{Markup}

\command{\latexcmd{textbf}{Bold}}{\textbf{Bold}}
\command{\latexcmd{textit}{Italics}}{\textit{Italics}}
\command{\latexcmd{textsl}{Slanted}}{\textsl{Slanted}}
\command{\latexcmd{textsc}{Small caps}}{\textsc{Small caps}}
\command{\latexcmd{underline}{Underline}}{\underline{Underline}}

\sectiontitle{Lists and floats}

\command{\environment{itemize}}{Bulleted list}
\command{\environment{enumerate}}{Numbered list}\
\command{\environment{description}}{Description list}


%----------------------------------------------------------------------------------------

\end{minipage} % End the first column of text
} % End the first division of the page

%----------------------------------------------------------------------------------------
%	SECOND COLUMN SPECIFICATION 
%----------------------------------------------------------------------------------------

\put(100,180){ % Divide the page
\begin{minipage}[t]{95mm} % Create a box to house text

%----------------------------------------------------------------------------------------
%	HEADING FOUR
%----------------------------------------------------------------------------------------
\command{\texttt{\textbackslash item}}{List item}
\command{\environment{figure}}{\texttt{figure} environment}
\command{\latexcmd{includegraphics}{<path>}}{Adds a picture}
\command{\latexcmd{caption}{Caption}}{Caption}
\command{\environment{table}}{Tabular container}
\command{\environment{tabular}}{Creates a table}
\command{\latexcmd{label}{<label>}}{Creates a label}
\command{\latexcmd{ref}{<label>}}{References a fig/tab/eq}
%----------------------------------------------------------------------------------------
%	HEADING FIVE
%----------------------------------------------------------------------------------------				
\sectiontitle{Math} % Heading five

\command{\latexcmd{usepackage}{amsmath}}{Loads \texttt{amsmath} package}
\command{\texttt{\$\dots\$} or \texttt{\textbackslash(\dots \textbackslash)}}{Inline math mode}
\command{\texttt{\$\$\dots\$\$} or \texttt{\textbackslash[\dots \textbackslash ]}}{Display math mode}
\command{\environment{equation}}{\texttt{equation} environment}
\command{\environment{gather}}{\texttt{gather} environment}
\command{\environment{align}}{\texttt{align} environment}

\sectiontitle{Chemistry}

\command{\latexcmd{usepackage}{mhchem}}{Loads \texttt{mhchem} package}
\command{\latexcmd{ce}{C6H12O6}}{\ce{C6H12O6}}
\command{\texttt{\$\textbackslash to\$}}{$\to$}
\command{\texttt{\$\textbackslash rightleftharpoons\$}}{$\rightleftharpoons$}

\sectiontitle{R integration (knitr/Sweave)}
\command{\texttt{\textless{}\textless{}options\textgreater{}\textgreater{}== ... @}}{R code block}
\command{\texttt{echo = TRUE/FALSE}}{Show R code in the document?}
\command{\texttt{eval = TRUE/FALSE}}{Evaluate the code?}

%----------------------------------------------------------------------------------------

\end{minipage} % End the second column of text
} % End the second division of the page

%----------------------------------------------------------------------------------------
%	THIRD COLUMN SPECIFICATION 
%----------------------------------------------------------------------------------------

\put(200,180){ % Divide the page
\begin{minipage}[t]{90mm} % Create a box to house tex

\command{\texttt{include = TRUE/FALSE}}{Show code and output?}
\command{\texttt{message = TRUE/FALSE}}{Show messages?}
\command{\texttt{warning = TRUE/FALSE}}{Show warnings?}

%----------------------------------------------------------------------------------------
%	FOOTNOTE
%----------------------------------------------------------------------------------------

\vspace{\baselineskip}
\linethickness{0.5mm} % Thickness of the footer line
{\color{mygray}\line(1,0){30}} % Print the line with a custom color

\footnotesize{
				
This template is released under the MIT license.
}

%----------------------------------------------------------------------------------------

\end{minipage} % End the third column of text
} % End the third division of the page
\end{picture} % End the container for the entire page

%----------------------------------------------------------------------------------------

\end{document}